\documentclass[
aps,%
12pt,%
final,%
notitlepage,%
oneside,%
onecolumn,%
nobibnotes,%
nofootinbib,% 
superscriptaddress,%
noshowpacs,%
amsmath,%
amssymb,%
centertags]%
{revtex4-2}

\usepackage[T2A]{fontenc}
\usepackage[utf8]{inputenc}
\usepackage[russian]{babel}

\usepackage{graphicx}

\newcommand*{\No}{\textnumero}

\begin{document}

\title{НАЗВАНИЕ СТАТЬИ,\\
РАЗБИТОЕ НА СТРОКИ}% Разбиение на строки осуществляется командой \\

\author{\firstname{П.~A.}~\surname{Первый-Автор}}
% Здесь разбиение на строки осуществляется автоматически или командой \\
\email{First.Author@institution.ras.ru}
\affiliation{%
Место работы и/или адрес первого и второго авторов
}%
\author{\firstname{В.~А.}~\surname{Второй-Автор}}
\email{Second.Author@institution.ras.ru}
\affiliation{%
Место работы и/или адрес первого и второго авторов
}%

\author{\firstname{Т.~А.}~\surname{Третий-Автор}}
\email{Third.Author@univ.edu}
%\noaffiliation % если у автора место работы не указывается
\affiliation{%
Место работы и/или адрес третьего автора
}%


\author{\firstname{Ч.~А.}~\surname{Четвертый-Автор}}
\email{Fourth.Author@inst.ras.ru}
\affiliation{%
Место работы и/или адрес четвертого автора
}%

\begin{abstract}
В этом примере статьи содержатся некоторые необходимые автору
сведения и примеры того, как набрать статью в REV\TeX~4 для
журналов, издаваемых Международной академической издательской
компанией \flqqНаука/Интерпериодика\frqq.
\end{abstract}

\maketitle

\section{Введение}

Процесс оформлении статьи с помощью REV\TeX~4 подробно описан в
руководстве по работе с REV\TeX~4 \cite{RTeX}. Большую помощь в
разрешениии возникших \TeX{}'нических вопросов могут оказать книги
\cite{GG,L}.

Для набора кавычек \glqqлапок\grqq используйте команды
\verb|\glqq| и \verb|\grqq|, а для кавычек \flqq{}елочек\frqq ---
\verb|\flqq| и \verb|\frqq|. Описание этих и других макросов
зависящих от русского языка можно найти в \cite{babel} страницы
29-30. Знак номера \No{} вводится командой \verb|\No|.

Для статей на русском языке используется стандартная русификация
и LH-шрифты, включенная в состав пакета \LaTeXe. Кодировка
\TeX{}-файлов CP866 (альтернативная).

Особенностью пакета REV\TeX~4 является использование rty-файла
(см. \cite{RTeX}, стр. 15, раздел "Job Macro Package"). В файле
maik.rty подключаются в правильном порядке необходимые пакеты
LaTex 2e, пeреопределяются некоторые команды REVTeX 4 и LaTeX 2e,
связанные с оформлением русского варианта статьи в МАИК. Не
желательно использование пакетов LaTeX 2e отличных от включенных
в maik.rty. Не вносите никаких изменений в файл maik.rty. Файл
maik.rty необходимо разместить в директории доступной для поиска
TeX-компилятору.


\nocite{*}

\bibliography{main}

\end{document}